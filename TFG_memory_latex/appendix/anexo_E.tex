\chapter{Pruebas de concepto y MVP}\label{S:anexo_E}

En este anexo se citas las principales y mas útiles elementos de pruebas de concepto así como Mínimo Producto Viable MVP basico de cara a usar en proyecto o necesidades de una pequeña empresa.

\section{Pruebas de Concepto}\label{S:pruebas_concepto}
Las pruebas de conceptos, son docker-compose en donde se instancian los servicios auxiliares y servicios a probar. Aunque no requiere de las restricciones o interconexiones con otras redes usualmente se tiene a definir redes internas de docker si es necesario así como volúmenes en las mismas carpeta del docker-compose. Muchas de estas pruebas se basan en la versión oficial o ejemplos proveídos por su principal comunidad mantenedora.

\subsection{Comunicación}
Existen multitud de servicios de comunicación revisadas incluso probados durante la realización de este documento tabla \ref{T:servicios_dockerizados}. En muchos casos especialmente en los relacionados con email la configuración es bastante compleja o en muchos sistemas similares a slack usan demasiados recursos o subsistemas auxiliares. Uno de los sistemas mas aceptables de ser usado es rocket chat\cite{c_rocket_chat} con una base de datos auxiliar en este caso mongodb.

\begin{lstlisting}[style=yaml, caption={docker-compose.yml Rocket Chat prueba de concepto.}, label={lst:rocket_chat} ]
version: '3.3'
services:  
  rocketchat:
    image: registry.rocket.chat/rocketchat/rocket.chat:6.1.0
    restart: unless-stopped
    volumes:
      - rocketchat-uploads:/app/uploads
    environment:
      MONGO_URL: mongodb://mongodb:27017/rocketchat?replicaSet=rs0
      MONGO_OPLOG_URL: mongodb://mongodb:27017/local?replicaSet=rs0
      ROOT_URL: http://localhost:3000    # https://rocketchat.example.com
      PORT: 3000
      DEPLOY_METHOD: docker
      Accounts_UseDNSDomainCheck: 'false'
    depends_on:
      - mongodb
    expose:
      - 3000
    ports:
      - "3000:3000"

  mongodb:
    image: bitnami/mongodb:4.4
    restart: unless-stopped
    volumes:
      - mongodb_data:/bitnami/mongodb
    environment:
      MONGODB_REPLICA_SET_MODE: primary
      MONGODB_REPLICA_SET_NAME: rs0
      MONGODB_PORT_NUMBER: 27017
      MONGODB_INITIAL_PRIMARY_HOST: mongodb
      MONGODB_INITIAL_PRIMARY_PORT_NUMBER: 27017
      MONGODB_ADVERTISED_HOSTNAME: mongodb
      MONGODB_ENABLE_JOURNAL: 'true'
      ALLOW_EMPTY_PASSWORD: 'yes'

volumes:
  mongodb_data:
  rocketchat-uploads:    

\end{lstlisting}

\subsection{Ejemplo web}
La herramienta mas fácil de generar una web dinámica así como gestionarla es utilizar un framework, especialmente si hablamos de un framework que debe ser gestionado por personas no técnicas sin tocar código. Wordpress es una excelente opción, es mayoritario, existen una gran cantidad de plugins y módulos prefabricados y guías extensas que permiten crear una web rápida y fácilmente.

\begin{lstlisting}[style=yaml, caption={docker-compose.yml Wordpress prueba de concepto.}, label={lst:wordpress} ]
version: "3.5"
services:
#
# Wordpress blog page
#
  mariadb:
    container_name: mariadb
    image: docker.io/bitnami/mariadb:11.0
    ports:
      - '23306:3306'
    volumes:
      - './mariadb/:/bitnami/mariadb'
    environment:
      #ALLOW_EMPTY_PASSWORD: "yes"
      MARIADB_USER: ${MARIADB_USER:-bn_wordpress}
      MARIADB_PASSWORD: ${MARIADB_PASSWORD:-p4ssw0rd_wordpress}
      MARIADB_DATABASE: ${MARIADB_DATABASE:-bitnami_wordpress}
      MARIADB_ROOT_PASSWORD: ${MARIADB_ROOT_PASS:-root_pass}
      
  wordpress:
    container_name: wordpress
    image: docker.io/bitnami/wordpress:6
    ports:
      - '28080:8080'
      - '28443:8443'
    volumes:
      - './wordpress/:/bitnami/wordpress'
    depends_on:
      - mariadb
      - nginx
    environment:
      VIRTUAL_HOST: blog.programing.es
      VIRTUAL_PORT: 8080
      LETSENCRYPT_HOST: blog.programing.es
      WORDPRESS_DATABASE_HOST: mariadb
      WORDPRESS_DATABASE_PORT_NUMBER: 3306
      WORDPRESS_DATABASE_NAME: ${MARIADB_DATABASE:-bitnami_wordpress}
      WORDPRESS_DATABASE_USER: ${MARIADB_USER:-bn_wordpress}
      WORDPRESS_DATABASE_PASSWORD: ${MARIADB_PASSWORD:-p4ssw0rd_wordpress}
      WORDPRESS_USERNAME: ${WORDPRESS_USER:-ropnom}
      WORDPRESS_PASSWORD: ${WORDPRESS_PASSWORD:-tfg}
      WORDPRESS_EMAIL: ${WORDPRESS_EMAIL:-rodrigo.sc@programing.es}
      WORDPRESS_FIRST_NAME: ${WORDPRESS_FIRST_NAME:-Rodrigo}
      WORDPRESS_LAST_NAME: ${WORDPRESS_LAST_NAME:-Sampedro}
      WORDPRESS_BLOG_NAME: ${WORDPRESS_BLOG_NAME:-"Programar Ingenieria"}
      WORDPRESS_PLUGINS: ${WORDPRESS_PLUGINS}
      WORDPRESS_SMTP_HOST: ${WORDPRESS_SMTP_HOST}
      WORDPRESS_SMTP_PORT: ${WORDPRESS_SMTP_PORT}
      WORDPRESS_SMTP_USER: ${WORDPRESS_SMTP_USER}
      WORDPRESS_SMTP_PASSWORD: ${WORDPRESS_SMTP_PASSWORD}
      WORDPRESS_ENABLE_REVERSE_PROXY: "yes"
\end{lstlisting}

\subsection{Ejemplo wiki}
Para la evaluación del uso de wiki o documentación existen múltiples alternativas. Dependiendo de la casuística, documentación de software, pagina de documentación e información colectiva, foros de debate y reporte, existen múltiples herramientas aplicables. Con el fin de cubrir diversas casuísticas especialmente en nivel de recursos se han evaluado 3 opciones, bajos recursos para documentación PineDocs\cite{c_pinedocs}, intermedio en recursos y comunitario MediaWiki\cite{c_media_wiki}, y casos customizados para desarrollo sin usar excesivos recursos BookStack\cite{c_bookstack}.

\subsubsection{PineDocs}
\begin{lstlisting}[language=yml, style=yaml, caption={docker-compose.yml PineDocs prueba de concepto.}, label={lst:pinedocs} ]
version: '3'
services:
  web:
    image: xy2z/pinedocs:1.2.5
    ports:
      - 3000:80
    volumes:
      - ./data:/data/pinedocs
\end{lstlisting}

\subsubsection{MediaWiki}
\begin{lstlisting}[ style=yaml, caption={docker-compose.yml MediaWiki prueba de concepto.}, label={lst:mediawiki} ]
version: '3'
services:
  mediawiki:
    image: mediawiki
    restart: always
    ports:
      - 8080:80
    links:
      - database
    volumes:
      - ./images:/var/www/html/images
      # After initial setup, download LocalSettings.php to the same directory as
      # this yaml and uncomment the following line and use compose to restart
      # the mediawiki service
      #- ./LocalSettings.php:/var/www/html/LocalSettings.php
  # This key also defines the name of the database host used during setup instead of the default "localhost"
  database:
    image: mariadb
    restart: always
    environment:
      # @see https://phabricator.wikimedia.org/source/mediawiki/browse/master/includes/DefaultSettings.php
      MYSQL_DATABASE: my_wiki
      MYSQL_USER: wikiuser
      MYSQL_PASSWORD: example
      MYSQL_RANDOM_ROOT_PASSWORD: 'yes'
    volumes:
      - ./db:/var/lib/mysql
\end{lstlisting}

\subsubsection{BookStack}

\begin{lstlisting}[ style=yaml, caption={docker-compose.yml BookStack prueba de concepto.}, label={lst:bookstack} ]
version: "3"
services:
  bookstack:
    image: ghcr.io/linuxserver/bookstack
    container_name: bookstack
    environment:
      - PUID=1000
      - PGID=1000
      - TZ=Europe/Madrid  
      - APP_URL=http://localhost:6875   
      - DB_HOST=bookstack_db
      - DB_USER=bookstack
      - DB_PASS=p4ssw0rd
      - DB_DATABASE=bookstackapp
    volumes:
      - ./app-data:/config
    ports:
      - "6875:80"
    restart: unless-stopped
    depends_on:
      - bookstack_db

  bookstack_db:
    image: ghcr.io/linuxserver/mariadb
    container_name: bookstack_db
    environment:
      - PUID=1000
      - PGID=1000
      - TZ=Europe/Madrid    
      - MYSQL_ROOT_PASSWORD=p4ssw0rd
      - MYSQL_DATABASE=bookstackapp
      - MYSQL_USER=bookstack
      - MYSQL_PASSWORD=p4ssw0rd
    volumes:
      - ./db-data:/config
    restart: unless-stopped
\end{lstlisting}

\subsection{Cloud Storage}
Se han evaluado diferentes opciones de bajos y medianos recursos como servicios de almacenamiento de datos extendido y sincronizado (cloud).

\subsubsection{Sea File}
\begin{lstlisting}[ style=yaml, caption={docker-compose.yml SeaFile prueba de concepto.}, label={lst:sea_file} ]
version: '2.0'
services:
  db:
    image: mariadb:10.6
    container_name: seafile-mysql
    environment:
      - MYSQL_ROOT_PASSWORD=db_dev  # Requested, set the root's password of MySQL service.
      - MYSQL_LOG_CONSOLE=true
    volumes:
      - ./db:/var/lib/mysql  # Requested, specifies the path to MySQL data persistent store.
    networks:
      - seafile-net

  memcached:
    image: memcached:1.6.18
    container_name: seafile-memcached
    entrypoint: memcached -m 256
    networks:
      - seafile-net
          
  seafile:
    image: seafileltd/seafile-mc:latest
    container_name: seafile
    ports:
      - "80:80"
#     - "443:443"  # If https is enabled, cancel the comment.
    volumes:
      - ./seafile-data:/shared   # Requested, specifies the path to Seafile data persistent store.
    environment:
      - DB_HOST=db
      - DB_ROOT_PASSWD=db_dev  # Requested, the value shuold be root's password of MySQL service.
      - TIME_ZONE=Etc/UTC  # Optional, default is UTC. Should be uncomment and set to your local time zone.
      - SEAFILE_ADMIN_EMAIL=me@example.com # Specifies Seafile admin user, default is 'me@example.com'.
      - SEAFILE_ADMIN_PASSWORD=asecret     # Specifies Seafile admin password, default is 'asecret'.
      - SEAFILE_SERVER_LETSENCRYPT=false   # Whether to use https or not.
      - SEAFILE_SERVER_HOSTNAME=docs.seafile.com # Specifies your host name if https is enabled.
    depends_on:
      - db
      - memcached
    networks:
      - seafile-net

networks:
  seafile-net:
\end{lstlisting}

\subsubsection{Samba server}
\begin{lstlisting}[style=yaml, caption={docker-compose.yml Samba prueba de concepto.}, label={lst:samba} ]
version: '3.4'
services:
  samba:
    image: dperson/samba
    environment:
      TZ: 'EST5EDT'
    networks:
      - default
    ports:
      - "137:137/udp"
      - "138:138/udp"
      - "139:139/tcp"
      - "445:445/tcp"
    read_only: true
    tmpfs:
      - /tmp
    restart: unless-stopped
    stdin_open: true
    tty: true
    volumes:
      - /mnt:/mnt:z
      - /mnt2:/mnt2:z
    command: '-s "Mount;/mnt" -s "Bobs Volume;/mnt2;yes;no;no;bob" -u "bob;bobspasswd" -p'

networks:
  default:
\end{lstlisting}

\subsubsection{NextCloud}
\begin{lstlisting}[ style=yaml, caption={docker-compose.yml NextCloud prueba de concepto.}, label={lst:nextcloud} ]
version: "3.4"
#https://gist.github.com/mrzapp/08947dd861bc826c4e02cee2d4da51ae
networks: 
    postgres: ~
    redis: ~

services:
    # PostgreSQL
    postgres: 
        container_name: postgres
        environment: 
            - POSTGRES_PASSWORD=somepassword
            - POSTGRES_USER=someuser
        image: postgres:latest
        restart: always
        volumes: 
            - "./postgres/init:/docker-entrypoint-initdb.d/"
            - "./postgres/data:/var/lib/postgresql/data"
            - "/etc/localtime:/etc/localtime:ro"
        networks: 
            - postgres

    # NextCloud
    nextcloud: 
        container_name: nextcloud
        depends_on: 
          - postgres
        image: nextcloud:latest
        restart: always
        ports:
            - 1000:80
        volumes: 
            - "./nextcloud/data:/var/www/html/data"
            - "./nextcloud/config:/var/www/html/config"
            - "./nextcloud/themes:/var/www/html/themes"
            - "./nextcloud/custom_apps:/var/www/html/custom_apps"
            - "/etc/localtime:/etc/localtime:ro"
        networks: 
            - postgres
            - redis

    # Redis
    redis:
        container_name: redis
        image: redis:latest
        restart: always
        networks:
            - redis

    # Collabora
    collabora:
        container_name: collabora
        image: collabora/code:latest
        cap_add: 
            - MKNOD
        environment: 
            - domain=test\.local:9980
            - username=someuser
            - password=somepassword
        ports:
            - 9980:9980
        restart: always
        volumes:
            - "/etc/localtime:/etc/localtime:ro"
\end{lstlisting}

\subsection{Reverse Proxy y https}\label{S:reverse_proxy_example}
Aunque se ha evaluado Traefik\cite{c_traefik} y Ngix\cite{c_ngix}, pero debido a una mayor cantidad de soporte y documentación, se ha entendido que Ngix es mas apropiado como prueba de concepto resolutiva.

\begin{lstlisting}[style=yaml, caption={docker-compose.yml Ngix proxy y Let's Encrypt prueba de concepto.}, label={lst:https_ngix} ]
version: "3.5"
services:
  nginx:
    container_name: nginx
    image: nginxproxy/nginx-proxy
    restart: unless-stopped
    labels: 
      - com.github.nginx-proxy.nginx
    ports:
      - 80:80
      - 443:443
    volumes:
      - /var/run/docker.sock:/tmp/docker.sock:ro
      - ./nginx/html:/usr/share/nginx/html
      - ./nginx/certs:/etc/nginx/certs
      - ./nginx/vhost:/etc/nginx/vhost.d
    logging:
      options:
          max-size: "10m"
          max-file: "3"

  letsencrypt-companion:
    container_name: letsencrypt-companion
    image: jrcs/letsencrypt-nginx-proxy-companion
    restart: unless-stopped
    volumes:
      - /var/run/docker.sock:/var/run/docker.sock
      - ./nginx/html:/usr/share/nginx/html
      - ./nginx/certs:/etc/nginx/certs
      - ./nginx/vhost:/etc/nginx/vhost.d
      - ./nginx/acme:/etc/acme.sh
    environment:
      DEFAULT_EMAIL: ropnom5291@gmail.com
      NGINX_PROXY_CONTAINER: nginx

  hello-world:
    container_name: hello-world
    image: kornkitti/express-hello-world
    expose:
      - "8080"
    environment:
      VIRTUAL_HOST: test.programing.es
      LETSENCRYPT_HOST: test.programing.es
\end{lstlisting}

\subsection{Ticketing}
Después de evaluar diferentes softwares dockerizados de ticketing,he llegado a la conclusión que la gran mayoría de servicios externos gratuitos son bastante adecuados para  un equipo reducido.

Por otra parte la mayor parte de servicios de ticketing son muy costosos en recursos por lo que el único ejemplo valido bajo en recursos es Planka\cite{c_planka}.

\begin{lstlisting}[style=yaml, caption={docker-compose.yml Planka prueba de concepto.}, label={lst:planka} ]
version: '3'
services:
  planka:
    image: ghcr.io/plankanban/planka:latest
    command: >
      bash -c
        "for i in `seq 1 30`; do
          ./start.sh &&
          s=$$? && break || s=$$?;
          echo \"Tried $$i times. Waiting 5 seconds...\";
          sleep 5;
        done; (exit $$s)"
    restart: unless-stopped
    volumes:
      - ./user-avatars:/app/public/user-avatars
      - ./project-background-images:/app/public/project-background-images
      - ./attachments:/app/private/attachments
    ports:
      - 3000:1337
    environment:
      - BASE_URL=http://localhost:3000
      - TRUST_PROXY=0
      - DATABASE_URL=postgresql://postgres@postgres/planka
      - SECRET_KEY=notsecretkey

      # Can be removed after installation
      - DEFAULT_ADMIN_EMAIL=demo@demo.demo # Do not remove if you want to prevent this user from being edited/deleted
      - DEFAULT_ADMIN_PASSWORD=demo
      - DEFAULT_ADMIN_NAME=Demo Demo
      - DEFAULT_ADMIN_USERNAME=demo

      # related: https://github.com/knex/knex/issues/2354
      # As knex does not pass query parameters from the connection string we
      # have to use environment variables in order to pass the desired values, e.g.
      # - PGSSLMODE=<value>

      # Configure knex to accept SSL certificates
      # - KNEX_REJECT_UNAUTHORIZED_SSL_CERTIFICATE=false
    depends_on:
      - postgres

  postgres:
    image: postgres:14-alpine
    restart: unless-stopped
    volumes:
      - ./db-data:/var/lib/postgresql/data
    environment:
      - POSTGRES_DB=planka
      - POSTGRES_HOST_AUTH_METHOD=trust


\end{lstlisting}

\subsection{SSO Ldap y Keycloak}\label{S:sso_example}
Con el fin de implementar un SSO, se ha utilizado LDAP\cite{c_ldap} y una interfaz gráfica de configuración y un servicio de Keycloak\cite{c_keycloak}.

\begin{lstlisting}[ style=yaml, caption={docker-compose.yml SSO LDAP-Keycloak prueba de concepto.}, label={lst:sso_keycloak} ]
version: '3.8'
services:      
  mysql:
    image: mysql:5.7.43
    container_name: mysql
    ports:
      - "3306:3306"
    environment:
      - MYSQL_DATABASE=keycloak
      - MYSQL_USER=keycloak
      - MYSQL_PASSWORD=password
      - MYSQL_ROOT_PASSWORD=root_password
    healthcheck:
      test: "mysqladmin ping -u root -p$${MYSQL_ROOT_PASSWORD}"

  keycloak:
    image: quay.io/keycloak/keycloak:22.0.3
    container_name: keycloak
    environment:
      - KEYCLOAK_ADMIN=admin
      - KEYCLOAK_ADMIN_PASSWORD=admin
      - KC_DB=mysql
      - KC_DB_URL_HOST=mysql
      - KC_DB_URL_DATABASE=keycloak
      - KC_DB_USERNAME=keycloak
      - KC_DB_PASSWORD=password
      - KC_HEALTH_ENABLED=true
    ports:
      - "8080:8080"
    command: start-dev
    depends_on:
      - mysql
    healthcheck:
      test: "curl -f http://localhost:8080/health/ready || exit 1"
     
  openldap:
    image: osixia/openldap:1.5.0
    container_name: openldap
    environment:
      - LDAP_ORGANISATION="MyCompany Inc."
      - LDAP_DOMAIN=mycompany.com
    ports:
      - "389:389"

  phpldapadmin:
    image: osixia/phpldapadmin:0.9.0
    container_name: phpldapadmin
    environment:
      - PHPLDAPADMIN_LDAP_HOSTS=openldap
    ports:
      - "6443:443"
    depends_on:
      - openldap
\end{lstlisting}

\subsection{VPN}
En el caso de VPN se han evaluado las dos opciones mas utilizadas OpenVPN y wireguard, sin embargo se han descartado  los clientes multiprotocolo y comerciales por opciones ligeras, eficientes e integradas con una UI simple.

\subsubsection{OpenVPN}
Se ha buscado una versión especialmente ligera de openvpn y una interfaz web sencilla y fácil de gestionar:
\begin{lstlisting}[ style=yaml, caption={docker-compose.yml OpenVPN con UI.}, label={lst:openvpn} ]
---
version: "3.5"
#define networks
networks:
  openvpn:
    name: openvpn
    #external: true
    internal: false
    driver: bridge
    driver_opts:
      com.docker.network.bridge.name: openvpn
    ipam:
      config:
        - subnet: 192.168.88.0/24
services:
    openvpn:
       container_name: openvpn
       image: d3vilh/openvpn-server:latest-amd64
       privileged: true
       ports: 
          - "1194:1194/udp"
       environment:
           TRUST_SUB: 10.0.70.0/24
           GUEST_SUB: 10.0.71.0/24
           HOME_SUB: 192.168.88.0/24
       volumes:
           - ./pki:/etc/openvpn/pki
           - ./clients:/etc/openvpn/clients
           - ./config:/etc/openvpn/config
           - ./staticclients:/etc/openvpn/staticclients
           - ./log:/var/log/openvpn
           - ./fw-rules.sh:/opt/app/fw-rules.sh
       cap_add:
           - NET_ADMIN
       networks:
        - openvpn
       restart: always

    openvpn-ui:
       container_name: openvpn-ui
       image: d3vilh/openvpn-ui:latest-amd64
       environment:
           - OPENVPN_ADMIN_USERNAME=admin
           - OPENVPN_ADMIN_PASSWORD=gagaZush
       privileged: true
       ports:
           - "8080:8080/tcp"
       volumes:
           - ./:/etc/openvpn
           - ./db:/opt/openvpn-ui/db
           - ./pki:/usr/share/easy-rsa/pki
           - /var/run/docker.sock:/var/run/docker.sock:ro
       restart: always
       networks:
        - openvpn
\end{lstlisting}

\subsubsection{Wireguard}
En el caso de wireguard se han encontrado dos soluciones alternativas que han sido utilizadas para crear el ejemplo de múltiples layer de vpn.

\subsubsection{Wireguard UI}
\begin{lstlisting}[style=yaml, caption={docker-compose.yml Wireguard prueba de concepto.}, label={lst:wireguard-ui} ]
version: "3.8"
networks:
  external_net:
    external: true
    name: external_net
services:
  # WireGuard VPN service
  wireguard:
    image: linuxserver/wireguard:latest
    container_name: wireguard
    cap_add:
      - NET_ADMIN
      - SYS_MODULE
    volumes:
      - ./config:/config
    ports:
      # Port of the WireGuard VPN server
      - 51820:51820/udp

  # WireGuard-UI service
  wireguard-ui:
    image: ngoduykhanh/wireguard-ui:latest
    container_name: wireguard-ui
    depends_on:
      - wireguard
    cap_add:
      - NET_ADMIN
    # Use the network of the 'wireguard' service
    # This enables to show active clients in the status page
    #network_mode: service:wireguard
    environment:
      - SENDGRID_API_KEY
      - EMAIL_FROM_ADDRESS
      - EMAIL_FROM_NAME
      - SESSION_SECRET
      - WGUI_USERNAME=admin
      - WGUI_PASSWORD=password
      - WG_CONF_TEMPLATE
      - WGUI_MANAGE_START=true
      - WGUI_MANAGE_RESTART=true
    logging:
      driver: json-file
      options:
        max-size: 50m
    volumes:
      - ./db:/app/db
      - ./config:/etc/wireguard
    ports:
      # Port for WireGuard-UI
      - 5000:5000
  http:
      image: nginxdemos/hello:latest
      labels:
        SERVICE_80_NAME: http
        SERVICE_TAGS: production
      ports:
        - 8080:80
\end{lstlisting}

\subsubsection{Wireguard Easy}
\begin{lstlisting}[style=yaml, caption={docker-compose.yml Wireguard Easy prueba de concepto.}, label={lst:wg-easy} ]
version: "3.5"
networks:
  external_front:
    name: external_front
    #external: true
    internal: false
    driver: bridge
    driver_opts:
      com.docker.network.bridge.name: docker_external
    ipam:
      config:
        - subnet: 192.168.99.0/24
services:
  wg-easy:
    environment:
      - WG_HOST=programing.es
      # Optional:
      - PASSWORD=pass
      # - WG_PORT=21821
      - WG_DEFAULT_ADDRESS=10.252.1.x
      # - WG_DEFAULT_DNS=1.1.1.1
      # - WG_MTU=1420
      # - WG_ALLOWED_IPS=192.168.15.0/24, 10.0.1.0/24
      # - WG_PRE_UP=echo "Pre Up" > /etc/wireguard/pre-up.txt
      # - WG_POST_UP=echo "Post Up" > /etc/wireguard/post-up.txt
      # - WG_PRE_DOWN=echo "Pre Down" > /etc/wireguard/pre-down.txt
      # - WG_POST_DOWN=echo "Post Down" > /etc/wireguard/post-down.txt
      
    image: weejewel/wg-easy
    container_name: wg-easy
    volumes:
      - .config:/etc/wireguard
    ports:
      - "51820:51820/udp" # 51820/udp"
      - "51821:51821/tcp"
    restart: unless-stopped
    cap_add:
      - NET_ADMIN
      - SYS_MODULE
    sysctls:
      - net.ipv4.ip_forward=1
      - net.ipv4.conf.all.src_valid_mark=1
    networks:
      - external_front

  http2:
    image: nginxdemos/hello:latest
    labels:
      SERVICE_80_NAME: http
      SERVICE_TAGS: production
    hostname: demo2.internal
    # ports:
    #   - 80
    networks:
      - external_front
\end{lstlisting}

\subsection{CI / CD}\label{S:ci_ejemplo}
Se han realizado diferentes pruebas de CI/CD especialmente con jenkins sin embargo el ejemplo mas simple y dockerizable con bajos recursos es drone, así mismo se ha utilizado gitea\cite{c_gitea} como principal repositorio de control de versiones.

\begin{lstlisting}[style=yaml, caption={docker-compose.yml Gitea-Drone CI prueba de concepto.}, label={lst:ci_gitea_drone} ]
version: '3.6'
services:
  gitea:
    container_name: gitea
    image: gitea/gitea:${GITEA_VERSION:-1.20}
    restart: unless-stopped
    environment:
      # https://docs.gitea.io/en-us/install-with-docker/#environments-variables
      - APP_NAME="Gitea"
      - USER_UID=1000
      - USER_GID=1000
      - RUN_MODE=prod
      - DOMAIN=${IP_ADDRESS}
      - SSH_DOMAIN=${IP_ADDRESS}
      - HTTP_PORT=3000
      - ROOT_URL=http://${IP_ADDRESS}:3000
      - SSH_PORT=222
      - SSH_LISTEN_PORT=22
      - DB_TYPE=sqlite3
    ports:
      - "3000:3000"
      - "222:22"
    networks:
      - cicd_net
    volumes:
      - ./gitea:/data

  drone:
    container_name: drone
    image: drone/drone:${DRONE_VERSION:-2.20}
    restart: unless-stopped
    depends_on:
      - gitea
    environment:
      # https://docs.drone.io/server/provider/gitea/
      - DRONE_DATABASE_DRIVER=sqlite3
      - DRONE_DATABASE_DATASOURCE=/data/database.sqlite
      - DRONE_GITEA_SERVER=http://${IP_ADDRESS}:3000/
      - DRONE_GIT_ALWAYS_AUTH=false
      - DRONE_RPC_SECRET=${DRONE_RPC_SECRET}
      - DRONE_SERVER_PROTO=http
      - DRONE_SERVER_HOST=${IP_ADDRESS}:3001
      - DRONE_TLS_AUTOCERT=false
      - DRONE_USER_CREATE=${DRONE_USER_CREATE}
      - DRONE_GITEA_CLIENT_ID=${DRONE_GITEA_CLIENT_ID}
      - DRONE_GITEA_CLIENT_SECRET=${DRONE_GITEA_CLIENT_SECRET}
    ports:
      - "3001:80"
      - "9001:9000"
    networks:
      - cicd_net
    volumes:
      - /var/run/docker.sock:/var/run/docker.sock
      - ./drone:/data

  drone-runner:
    container_name: drone-runner
    image: drone/drone-runner-docker:${DRONE_RUNNER_VERSION:-1.8}
    restart: unless-stopped
    depends_on:
      - drone
    environment:
      # https://docs.drone.io/runner/docker/installation/linux/
      # https://docs.drone.io/server/metrics/
      - DRONE_RPC_PROTO=http
      - DRONE_RPC_HOST=drone
      - DRONE_RPC_SECRET=${DRONE_RPC_SECRET}
      - DRONE_RUNNER_NAME="${HOSTNAME}-runner"
      - DRONE_RUNNER_CAPACITY=2
      - DRONE_RUNNER_NETWORKS=cicd_net
      - DRONE_DEBUG=false
      - DRONE_TRACE=false
    ports:
      - "3002:3000"
    networks:
      - cicd_net
    volumes:
      - /var/run/docker.sock:/var/run/docker.sock

networks:
  cicd_net:
    name: cicd_net

\end{lstlisting}

\section{Wireguard LAN-VPN-LAN}
Se ha seleccionado wireguard como vpn principal, sin embargo la configuración de wireguard tanto en servidor como en cliente se basa en script estáticos de re-direccionamiento y enrutado para las diferentes LAN, redes docker internas y VPN.

En este apartado se muestra ejemplo de configuración de ambos, servidor y clientes par que el usuario a traves de la UI, pueda configurar manual y estáticamente sus decisiones.

\subsection{Nat masquerade hace red docker}
Para permitir la comunicación entre la red docker interna principal y la red VPN se puede utilizar un nat masquerada como indica la siguiente configuración de de ejemplo:

\begin{lstlisting}[language=bash, caption={Ejemplo de configuracion servidor Nat masquerade}, label={lst:wireguard_server_nat} ]
[Interface]
Address = 10.10.1.0/24 # vpn range
ListenPort = 51820 # vpn server port
PrivateKey = <private key> # vpn server private key
MTU = 1450 # vpn MTU size (1500 - overhead bytes encapsulated)
PostUp = iptables -A FORWARD -i %i -j ACCEPT; iptables -A FORWARD -o %i -j ACCEPT; iptables -t nat -A POSTROUTING -s 10.10.1.0/24 -o eth0 -j MASQUERADE; iptables -A INPUT -p udp -m udp --dport 21820 -j ACCEPT
PostDown = iptables -D FORWARD -i %i -j ACCEPT; iptables -D FORWARD -o %i -j ACCEPT; iptables -t nat -D POSTROUTING -s 10.10.1.0/24 -o eth0 -j MASQUERADE; iptables -D INPUT -p udp -m udp --dport 21820 -j ACCEPT
Table = auto
\end{lstlisting}

\subsection{Script en cliente}
Uno de los  punto manuales en este documentos, es la configuración de un elemento de la RED VPN y LAN como nodo intermedio de comunicaciones.
Debido a la naturaleza del elemento normalmente router portable o una rasberry pi, requiere en todo momento de:
\begin{itemize}
    \item Tener el ip forwarding activado para permitir el enrutado de paquetes entre redes.
    \item Definir su tabla de rutas, añadiendo manualmente que otras Redes permite enrutar y por que elemento. 
    \item Incluir la configuración pertinente de su IP en la VPN y rango de su LAN para ser ofertada en otros elementos de la red
\end{itemize}

En un proceso iterativo, es decir, una vez definida el 'mapa' de las redes que vamos a interconectar y que elementos participan en dicha interconexión, es necesario agrupar el conjunto de LAN a ofrecerse al resto de clientes VPN, y manipular dicha configuración para obviar las redes LAN en los elementos interconectadores.

Debido a la naturaleza de Wireguar UI, no es un mecanismo automatizado y debe hacerse a mano cliente a cliente. De igual manera aunque los elementos interconectores funcionen y la configuración de la VPN, es necesario editar manualmente los routers de las LAN para que indiquen adecuadamente a través de que elementos se puede acceder tanto ala VPN como al resto de LAN.

\begin{lstlisting}[language=bash, caption={Ejemplo de configuracion servidor para homenets}, label={lst:wireguard_LAN_config} ]
[Interface]
PrivateKey = <private key>
Address = <client ip>
DNS = <internal dns proxy relay>
PostUp = DROUTE=$(ip route | grep default | awk '{print $3}'); HOMENET=192.168.0.0/16; HOMENET2=10.0.0.0/8; HOMENET3=172.16.0.0/12; ip route add $HOMENET3 via $DROUTE;ip route add $HOMENET2 via $DROUTE; ip route add $HOMENET via $DROUTE;iptables -I OUTPUT -d $HOMENET -j ACCEPT;iptables -A OUTPUT -d $HOMENET2 -j ACCEPT; iptables -A OUTPUT -d $HOMENET3 -j ACCEPT;  iptables -A OUTPUT ! -o %i -m mark ! --mark $(wg show %i fwmark) -m addrtype ! --dst-type LOCAL -j REJECT
PreDown = HOMENET=192.168.0.0/16; HOMENET2=10.0.0.0/8; HOMENET3=172.16.0.0/12; ip route del $HOMENET3 via $DROUTE;ip route del $HOMENET2 via $DROUTE; ip route del $HOMENET via $DROUTE; iptables -D OUTPUT ! -o %i -m mark ! --mark $(wg show %i fwmark) -m addrtype ! --dst-type LOCAL -j REJECT; iptables -D OUTPUT -d $HOMENET -j ACCEPT; iptables -D OUTPUT -d $HOMENET2 -j ACCEPT; iptables -D OUTPUT -d $HOMENET3 -j ACCEPT
\end{lstlisting}

Aunque es factible automatizar con protocolos estándar OSPF o RIP, la interconexión automática y anunciado de estas redes, requiere de funcionalidades a típicas de un router LAN por lo que se han descartado esta opción ante la falta de soporte en muchos router habituales.

En el ejemplo \ref{lst:wireguard_LAN_config} podemos observar como el cliente cunado inicia la vpn wireguard, añade routas estáticas de destinación al vpn server, si estas son adecuadamente seteadas con las IP o elementos que interconectan las LAN, los clientes podran hacer peticiones a las LAN interconectadas por la VPN.


\section{Mínimo Viable Producto}
El MVP es una prueba de concepto que valida tanto las funcionalidades de servicios dockerizados en VPS, el rever proxy con https funciona, la generación de un dns interno y el uso de VPN para acceder a los servicios no públicos, así como una prueba de dos capas de VPN para servicios específicos.

Al ser un propósito general de prueba, muchos servicios web como pueden ser una wiki, un respositorio espejo etc se han sustituido pro meros mock web, que permiten visionar un mensaje de 'hello word' o propiedades del servidor web que los ejecuta.

\begin{lstlisting}[style=yaml, caption={docker-compose.yml MVP prueba de concepto.}, label={lst:mvp_mock} ]
version: "3.5"

#define networks
networks:
  external:
    name: z_external
    #external: true
    internal: false
    driver: bridge
    driver_opts:
      com.docker.network.bridge.name: external
    ipam:
      config:
        - subnet: 192.168.66.0/24
  vpn_layer1:
    name: vpn
    #external: true
    internal: false
    driver: bridge
    driver_opts:
      com.docker.network.bridge.name: vpn_layer1
    ipam:
      config:
        - subnet: 192.168.99.0/24
  vpn_layer2:
    name: depp_vpn
    #external: true
    internal: false
    driver: bridge
    driver_opts:
      com.docker.network.bridge.name: vpn_layer2
    ipam:
      config:
        - subnet: 192.168.100.0/24

services:

# ------------------
# Public services
# ------------------
# all manage by ngix-proxy and let's encrypt
#

  # reverse proxy service
  nginx:
    container_name: nginx-proxy
    image: nginxproxy/nginx-proxy
    restart: unless-stopped
    labels: 
      - com.github.nginx-proxy.nginx
    ports:
      - 80:80
      - 443:443
    volumes:
      - /var/run/docker.sock:/tmp/docker.sock:ro
      - ./nginx/html:/usr/share/nginx/html
      - ./nginx/certs:/etc/nginx/certs
      - ./nginx/vhost:/etc/nginx/vhost.d
    logging:
      options:
          max-size: "10m"
          max-file: "3"
    networks:
      - external

  # Let's encrypt service (tls certificates for https service)
  letsencrypt-companion:
    container_name: letsencrypt-companion
    image: jrcs/letsencrypt-nginx-proxy-companion
    restart: unless-stopped
    volumes:
      - /var/run/docker.sock:/var/run/docker.sock
      - ./nginx/html:/usr/share/nginx/html
      - ./nginx/certs:/etc/nginx/certs
      - ./nginx/vhost:/etc/nginx/vhost.d
      - ./nginx/acme:/etc/acme.sh
    environment:
      DEFAULT_EMAIL: ropnom5291@gmail.com
      NGINX_PROXY_CONTAINER: nginx
    networks:
      - external

#
# Wordpress blog page
#
  mariadb:
    container_name: mariadb
    image: docker.io/bitnami/mariadb:11.0
    volumes:
      - './mariadb/:/bitnami/mariadb'
    environment:
      #ALLOW_EMPTY_PASSWORD: "yes"
      MARIADB_USER: ${MARIADB_USER:-bn_wordpress}
      MARIADB_PASSWORD: ${MARIADB_PASSWORD:-p4ssw0rd_wordpress}
      MARIADB_DATABASE: ${MARIADB_DATABASE:-bitnami_wordpress}
      MARIADB_ROOT_PASSWORD: ${MARIADB_ROOT_PASS:-root_pass}
    networks:
      - external
      
  wordpress:
    container_name: wordpress_blog
    image: docker.io/bitnami/wordpress:6
    # ports:
    #   - '28080:8080'
    #   - '28443:8443'
    expose:
      - '8080'
    volumes:
      - './wordpress/:/bitnami/wordpress'
    depends_on:
      - mariadb
      - nginx
    environment:
      VIRTUAL_HOST: blog.programing.es
      VIRTUAL_PORT: 8080
      LETSENCRYPT_HOST: blog.programing.es
      WORDPRESS_DATABASE_HOST: mariadb
      WORDPRESS_DATABASE_PORT_NUMBER: 3306
      WORDPRESS_DATABASE_NAME: ${MARIADB_DATABASE:-bitnami_wordpress}
      WORDPRESS_DATABASE_USER: ${MARIADB_USER:-bn_wordpress}
      WORDPRESS_DATABASE_PASSWORD: ${MARIADB_PASSWORD:-p4ssw0rd_wordpress}
      WORDPRESS_USERNAME: ${WORDPRESS_USER:-ropnom}
      WORDPRESS_PASSWORD: ${WORDPRESS_PASSWORD:-tfg}
      WORDPRESS_EMAIL: ${WORDPRESS_EMAIL:-rodrigo.sc@programing.es}
      WORDPRESS_FIRST_NAME: ${WORDPRESS_FIRST_NAME:-Rodrigo}
      WORDPRESS_LAST_NAME: ${WORDPRESS_LAST_NAME:-Sampedro}
      WORDPRESS_BLOG_NAME: ${WORDPRESS_BLOG_NAME:-"Programar Ingenieria"}
      WORDPRESS_PLUGINS: ${WORDPRESS_PLUGINS}
      WORDPRESS_SMTP_HOST: ${WORDPRESS_SMTP_HOST}
      WORDPRESS_SMTP_PORT: ${WORDPRESS_SMTP_PORT}
      WORDPRESS_SMTP_USER: ${WORDPRESS_SMTP_USER}
      WORDPRESS_SMTP_PASSWORD: ${WORDPRESS_SMTP_PASSWORD}
      WORDPRESS_ENABLE_REVERSE_PROXY: "yes"
    networks:
      - external

#
# Other services like next cloud service
#

#
# VPN services
#
  # WireGuard VPN service
  wireguard:
    container_name: wireguard_vpn1
    image: linuxserver/wireguard:latest
    sysctls:
      - net.ipv4.ip_forward=1
      - net.ipv4.conf.all.src_valid_mark=1
    cap_add:
      - NET_ADMIN
      - SYS_MODULE
    volumes:
      - ./vpn/config:/config
    environment:
      VIRTUAL_HOST: vpn.programing.es
      LETSENCRYPT_HOST: vpn.programing.es
      VIRTUAL_PORT: 5000
    ports:
      # Port of the WireGuard VPN server
      - 21820:21820/udp
      # UI port
      # - 21500:5000
    networks:
      vpn_layer1:
      external:

  # WireGuard-UI service
  wireguard-ui:
    container_name: wireguard-ui-vpn1
    image: ngoduykhanh/wireguard-ui:latest
    depends_on:
      - wireguard
    cap_add:
      - NET_ADMIN
    environment:
      # SENDGRID_API_KEY:
      # EMAIL_FROM_ADDRESS:
      # EMAIL_FROM_NAME:
      # SESSION_SECRET:
      WGUI_USERNAME: admin
      WGUI_PASSWORD: pass
      # WG_CONF_TEMPLATE:
      WGUI_MANAGE_START: "true"
      WGUI_MANAGE_RESTART: "true"
    logging:
      driver: json-file
      options:
        max-size: 50m
    volumes:
      - ./vpn/ui/db:/app/db
      - ./vpn/config:/etc/wireguard
    # Use the network of the 'wireguard' service
    # This enables to show active clients in the status page
    network_mode: service:wireguard
    # ports: #using service:wireguard
    #   # Port for WireGuard-UI
    #   - 21500:5000
    # networks:
    #   - external

  wg-easy:
    image: weejewel/wg-easy
    container_name: wg-easy
    volumes:
      - ./config:/etc/wireguard
    # ports:
    #   - "51820:51820/udp"
    #   - "51821:51821/tcp"
    restart: unless-stopped
    environment:
      WG_HOST: vpn2.internal      
      PASSWORD: pass
      WG_PORT: 51820
      WG_DEFAULT_ADDRESS: 10.252.1.x
      WG_DEFAULT_DNS: 192.168.100.53
      WG_PERSISTENT_KEEPALIVE: 20
      WG_MTU: 1400
      WG_ALLOWED_IPS: "10.252.1.0/24,192.168.100.0/24"
      # - WG_PRE_UP=echo "Pre Up" > /etc/wireguard/pre-up.txt
      WG_POST_UP: "DETH=$$(ip route | grep default | awk '{print $$5}');iptables -A FORWARD -i %i -j ACCEPT; iptables -A FORWARD -o %i -j ACCEPT; iptables -t nat -A POSTROUTING -s 10.252.1.0/24 -o $$DETH -j MASQUERADE"
      # - WG_PRE_DOWN=echo "Pre Down" > /etc/wireguard/pre-down.txt
      WG_POST_DOWN: "DETH=$$(ip route | grep default | awk '{print $$5}');iptables -D FORWARD -i %i -j ACCEPT; iptables -D FORWARD -o %i -j ACCEPT; iptables -t nat -D POSTROUTING -s 10.252.1.0/24 -o $$DETH -j MASQUERADE"
    cap_add:
      - NET_ADMIN
      - SYS_MODULE
    sysctls:
      - net.ipv4.ip_forward=1
      - net.ipv4.conf.all.src_valid_mark=1
    networks:
      vpn_layer1:
        ipv4_address: 192.168.99.100
        aliases:
          - vpn2.internal
      vpn_layer2:

  # VPN dns service
  dns-proxy:
    container_name: dns-proxy_mageddo
    image: defreitas/dns-proxy-server:3.14.5
    # ports:
    #   - "192.168.99.53:53:53/udp"
    #   - "5380:5380"
    volumes:
      - /var/run/docker.sock:/var/run/docker.sock
      - /etc/resolv.conf:/etc/resolv.conf
      - ./dns/config/:/app/conf/
    environment:
      MG_REGISTER_CONTAINER_NAMES: "true"
    networks:
      vpn_layer1:
        ipv4_address: 192.168.99.53 #fix ip for relay dns
        aliases:
          - "dns1.internal"
      vpn_layer2:
        ipv4_address: 192.168.100.53 #fix ip for relay dns
        aliases:
          - "dns2.internal"

  dns-proxy2:
    container_name: dns-proxy-dynamic
    image: carlsverre/damn-simple-dns-proxy:latest
    # ports:
    #   - "192.168.99.53:53:53/udp"
    networks:
        vpn_layer1:
          ipv4_address: 192.168.99.54 #fix ip for main dns
  

  # demo web
  http:
    image: nginxdemos/hello:latest
    labels:
      SERVICE_80_NAME: http
      SERVICE_TAGS: production
    hostname: demo.internal
    # ports:
    #   - 80
    networks:
      vpn_layer1:
        aliases:
          - webtest.internal
    depends_on:
      - dns-proxy
      - dns-proxy2

  http2:
    image: nginxdemos/hello:latest
    labels:
      SERVICE_80_NAME: http
      SERVICE_TAGS: production
    hostname: demo2.internal
    # ports:
    #   - 80
    networks:
      vpn_layer2:
        aliases:
          - webtest2.internal
    depends_on:
      - dns-proxy
      - dns-proxy2


  hello-world:
    container_name: hello-world
    image: kornkitti/express-hello-world
    expose:
      - "8080"
    environment:
      VIRTUAL_HOST: test.programing.es
      LETSENCRYPT_HOST: test.programing.es
    networks:
      - external


\end{lstlisting}

\section{Aprovisionamiento básico Ansible}\label{S:ansible_ejemplo}
Debido a restricciones de seguridad y privacidad, así como evitar futuros ataques a la plataforma de Elenkar únicamente se muestras los procesos de abastecimiento junto a segurización básica del VPS.

Véase código del proyecto \cite{c_code}.
